% !TeX program = lualatex

\documentclass[language=english]{phoebe}

\title{Phoebe's Official LaTeX Template \& Style Guide}
\author{Christoph Otte, Fabian Scheuermann}
\date{\today}
\email{phoebe@uni-heidelberg.de}
\keywords{Phoebe, LaTeX, Template}
\abstract{This document is part of the official LaTeX template for Phoebe articles. It provides our authors with rules and guidelines for setting up the layout of their article's manuscript.}

% load your own preamble
\usepackage{preamble}

% load file with bibliography
\addbibresource{references.bib}

\begin{document}

% ---------------------------------------
% Front matter
% ---------------------------------------

\maketitle

% ---------------------------------------
% Main body of the article
% ---------------------------------------

\section{General Instructions}

% If you are completely new to LaTeX you may first have a look at Overleaf's documentation \citep{overleaf-doc}.
In case you are completely new to LaTeX: A great resource for start learning LaTeX which has also many useful tutorials and nice tips even for advanced users is Overleaf's documentation. You will find an online version at \url{https://www.overleaf.com/learn}.

\subsection{Getting started}

Every article project constists of several files:
\begin{enumerate}
\item \lstinline{article.tex} --- The main file where you put your article's content. 
\item \lstinline{phoebe.cls} --- The LaTeX class must always be included.
\item \lstinline{references.bib} --- The file where you manage your bibliography.
\item \lstinline{preamble.sty} --- An empty file where you may include additional packages and define custom commands.
\item \lstinline{asset/} --- A folder where you store all your images.
\item \lstinline{fonts/} --- A necessary folder for all needed fonts.
\end{enumerate}

\noindent Please use LuaLaTeX to compile your document. A minimal working example is given below.

\begin{lstlisting}
% article.tex

\documentclass[language=english]{phoebe}

% define some variables
\title{My Article}
\author{Jane Doe}
\date{\today}
\email{phoebe@uni-heidelberg.de}
\keywords{Phoebe, LaTeX, Template}
\abstract{Your abstract}

 % load file with bibliography
\addbibresource{references.bib}

\begin{document}

\maketitle

% your text goes here.

\begingroup
\RaggedRight
\printbibliography
\endgroup

\end{document}
\end{lstlisting}

\noindent You may find a template for \lstinline{references.bib} in Sec.~\ref{sec:bibliography}.

\subsection{Article Length}

Your article should be 2--5 pages long including footnotes, captions and bibliography. If you need more space you might want to consider splitting up the content and submit two shorter articles.

\subsection{Abstract}

Every article needs a short abstract of about 250 words to provide readers with a short summary of the content.

\subsection{Additional LaTeX Packages}

The class covers only a couple of features. If you need more packages, simply add them to the preamble. Never modify the class file. In case of compilation errors please contact \href{mailto:phoebe@uni-heidelberg.de}{phoebe@uni-heidelberg.de}.

\section{Typography}

\subsection{Punctuation}

\subsection{New Paragraphs}

If you want to start a new paragraph, you could either use the \TeX\ command \lstinline[language={}]|\par| or leave a blank line (this has the same effect). The commonly used \lstinline|\\| has a different effect, depending on the environment and only starts a new line. It should therefore be avoided.

\subsection{Headings}

Phoebe only uses sections and subsections.

\subsection{Hyphen and Dash}

A hyphen (single \lstinline|-|) is used to combine words (e.g.~low-density). The en-dash (two \lstinline|--|) is slightly larger and is used to indicate ranges (e.g.~this month we publish 2--5 articles). The en-dash is identical in length to the minus sign. When in math mode, LaTeX will automatically use a minus sign when a single \lstinline|-| is used. em-dashes (\lstinline|---|) should not be used in the journal.

LaTeX assumes that a period marks the end of a sentence and as such puts a bit of extra space after it. This is wrong if the period is used in an abbreviation, e.g.\ \enquote{i.e.} To avoid this, place a space (\lstinline|e.g.\ |) after the period. In the previous example, the abbreviation period also marks the end of the sentence. In such case only one period is required.

\subsection{Quotation Marks}
The typography of quotation marks usually vary from country to country. However, the package \lstinline{csquotes} facilitates 

\begin{lstlisting}
The quick brown fox \enquote{Harvey} jumps over the lazy dog \enquote{Bruno}.
\end{lstlisting}
Depending on the language you chose in the class options at the very beginning, \lstinline{csquotes} will fill in their correct form used in the respective country: 
\begin{quote}
The quick brown fox \enquote{Harvey} jumps over the lazy dog \enquote{Bruno}.
\end{quote}

\subsection{Units}

Variables are set in italics (this happens automatically in math mode), however units are always roman (upright). They should be separated by a non-breaking space. For reciprocal units, use a superscript, e.g.\ $\si{\cm\per\s}$. Use either SI or cgs units.

Please make use of the \lstinline{siunitx} package (already loaded with this class) as it takes care of the aforementioned rules.
\begin{lstlisting}
R = \SI{8,3145}{\joule\per\mole\per\kelvin}
\end{lstlisting}
\begin{equation}
R= \SI{8,3145}{\joule\per\mole\per\kelvin}
\end{equation}
If you need a unit that is not already defined, you can define your own units like so\footnote{You might want to put those type of commands into \lstinline{preamble.sty}.}
\begin{lstlisting}
\DeclareSIUnit\parsec{pc}
\end{lstlisting}
The axis of plots must have the according unit. This should be written as $x\,/\,\si{\cm}$. Note that commonly found notation $x\,[\si{\cm}]$ is not acceptable. Square brackets denote the unit of a quantity (just like value is denoted by curly brackets), i.e.\ $x=\{x\}[x]$. Take the example $x=\SI{12}{\cm}$, i.e.~$[x]=\si{\cm}$. For more details, see section 7 of the \href{https://www.nist.gov/pml/special-publication-811/nist-guide-si-chapter-7-rules-and-style-conventions-expressing-values}{NIST Guide to the SI}.

\subsection{URLs and Hyperlinks}

Please use \lstinline{\url} for writing blank URLs, e.g.\ \url{https://phoebe.pubpub.org}, and \lstinline{\href} if you want to write a description instead, e.g.\ \href{https://phoebe.pubpub.org}{Phoebe}. 

\begin{lstlisting}
\url{phoebe.pubpub.org}
\href{phoebe.pubpub.org}{Phoebe}
\end{lstlisting}
Keep in mind that URLs without \lstinline{https://} won't usually work.

Sections, equations, figures and tables can be referenced by using \lstinline{label} and \lstinline{ref} commands.

\begin{lstlisting}
\section{Methods}
\label{sec:methods}

See \ref{sec:methods} for the methods.
\end{lstlisting}
The value inside \lstinline{label} is completely arbitrary, however, there are some conventions you might want to follow:

\begin{table}
	\centering
    \caption{Conventions for labelling objects in \LaTeX.}
	\begin{tabular}{ll}
		\toprule
		Convention & Object \\
		\midrule
		\lstinline|sec| & Section \\
		\lstinline|ssec| & Subsection \\
		\lstinline|eq| & Equation \\
		\lstinline|fig| & Figure \\
		\lstinline|tab| & Table \\
		\bottomrule
	\end{tabular}
\end{table}

\section{Math}

\subsection{Equations}

\begin{lstlisting}
\begin{equation}
\int_{-\infty}^{+\infty}e^{-x^2}\mathrm{d}x=\sqrt{\pi}
\end{equation}
\end{lstlisting}

\begin{equation}
\int_{-\infty}^{+\infty}e^{-x^2}\mathrm{d}x=\sqrt{\pi}
\end{equation}
If you like to show a short calculation with some steps in between, you may want to align these equations for better readability:

\begin{lstlisting}
\begin{align*}
\int_a^b\sin(x)\mathrm{d}x &= -\frac{1}{2}\left[e^{ib}-e^{ia}-\left(e^{-ib}-e^{-ia}\right)\right] \\
&= \frac{e^{ia}+e^{-ia}}{2}-\frac{e^{ib}+e^{-ib}}{2} \\
&= \cos(a)-\cos(b)
\end{align*}
\end{lstlisting}

\begin{align*}
\int_a^b\sin(x)\mathrm{d}x &= -\frac{1}{2}\left[e^{ib}-e^{ia}-\left(e^{-ib}-e^{-ia}\right)\right] \\
&= \frac{e^{ia}+e^{-ia}}{2} - \frac{e^{ib}+e^{-ib}}{2} \\
&= \cos(a) - \cos(b)
% \symbfit{B}(P) &= \frac{\mu_0}{4\pi}\int\frac{\symbfit{I}\times \hat{r}'}{r'^2}\mathrm{d}l \\ &= \frac{\mu_0}{4\pi}I\int\frac{\mathrm{d}\symbfit{l}\times\hat{r}'}{\hat{r}'^2}
\end{align*}
Equations should always be punctuated whenever they appear at the end of a sentence.

\subsection{Differentials}

Differentials are always set upright:

\begin{lstlisting}
\frac{\mathrm{d}f(x)}{\mathrm{d}x}
\end{lstlisting}

\begin{equation}
\frac{\mathrm{d}f(x)}{\mathrm{d}x}
\end{equation}

\subsection{Vectors}

Vectors are set in bold-italic, please don't use any arrows:

\begin{lstlisting}
\begin{equation}
    \nabla\times\symbfit{E} = -\frac{\partial \symbfit{B}}{\partial t}
\end{equation}
\end{lstlisting}

\begin{equation}
    \nabla\times\symbfit{E} = -\frac{\partial \symbfit{B}}{\partial t}
\end{equation}

\subsection{Multi-letter function names}

\begin{lstlisting}
\begin{equation}
    \log e^x = x
\end{equation}
\end{lstlisting}

\begin{equation}
    \log e^x = x
\end{equation}
If you want to define new function names you may want to add

\begin{lstlisting}
\DeclareMathOperator{\MyNewFunction}{MNF}
\end{lstlisting}
to the preamble of your document.

\subsection{Definition, Theorem, Proof}

\begin{lstlisting}
\theoremstyle{definition}
\newtheorem{definition}{Definition}
\theoremstyle{plain}
\newtheorem{theorem}{Theorem}
\newtheorem{lemma}{Lemma}

\begin{definition}
Let $f:\mathbb{R}\longrightarrow\mathbb{R}$ be a function.
\end{definition}

\begin{theorem}
$f$ is a function.
\end{theorem}

\begin{proof}
You get it, right?    
\end{proof}
\end{lstlisting}

\begin{definition}
Let $f:\mathbb{R}\longrightarrow\mathbb{R}$ be a function.
\end{definition}

\begin{theorem}
$f$ is a function.
\end{theorem}

\begin{proof}
You get it, right?    
\end{proof}

\subsection{Tensors and Indices}

\subsection{Cases}

\begin{lstlisting}
f(n) = \begin{cases}
  n/2  & n \text{ is even} \\
  3n+1 & n \text{ is odd}
\end{cases}
\end{lstlisting}

\begin{equation}
f(n) = \begin{cases}
  n/2  & n \text{ is even} \\
  3n+1 & n \text{ is odd}
\end{cases}\end{equation}

\subsection{Matrices}

\begin{lstlisting}
\begin{pmatrix}
  0 & 1 \\
  1 & 0
\end{pmatrix}
\end{lstlisting}

\begin{equation}
\begin{pmatrix}
  0 & 1 \\
  1 & 0
\end{pmatrix}
\end{equation}

\section{Figures and Tables}

We follow the convention that tables are placed at the top of a page while figures appear at the bottom.

\begin{lstlisting}
\begin{table}[t]
	\begin{center}
		\caption{Your caption goes here.}
		\setlength\defaultaddspace{0.5em}
		\begin{tabularx}{\columnwidth}{ll}
			\toprule
			Column 1 & Column 2 \\
			\midrule
			A & 1 \\
			B & 2 \\
			\bottomrule
		\end{tabularx}
	\end{center}
\end{table}
\end{lstlisting}

\begin{table}[t]
	\begin{center}
		\caption{Some Shakespeare's plays. }
		\setlength\defaultaddspace{0.5em}
		\begin{tabularx}{\columnwidth}{lXr}
			\toprule
			Title & Type & Year \\
			\midrule
			A Midsummer Night's Dream & Comedy & 1594 \\
			Romeo and Juliet & Tragedy & 1595 \\
			Much Ado About Nothing & Comedy & 1599 \\
			Hamlet & Tragedy & 1601 \\
			Henry VIII & History & 1613 \\
			\bottomrule
		\end{tabularx}
	\end{center}
\end{table}

\begin{table*}[t]
	\centering
	\caption{Some Bond Films with Sean Connery.}
	\label{tab:bond}
	\setlength\defaultaddspace{0.5em}
	\begin{tabularx}{\textwidth}{llllll}
		\toprule
		Film Title & Year & Villain & Bond Girl & Key Car \\
		\midrule
		Dr No  & 1962 & Doctor No & Ursula Andress $\cdot$ \textit{Honey Ryder} & Sunbeam Alpine \\
		From Russia With Love  & 1963 & Red Grant & Daniela Bianchi $\cdot$ \textit{Tatiana Romanova} & Bentley Mark IV \\
		Goldfinger  & 1964 & Goldfinger & Honor Blackman $\cdot$ \textit{Pussy Galore} & Aston Martin DB5 \\
		Thunderball  & 1965 & Emilio Largo & Claudine Auger $\cdot$ \textit{Domino} & Aston Martin DB5 \\
		You Only Live Twice  & 1967 & Blofeld & Akiko Wakabayashi $\cdot$ \textit{Aki} & Toyota 2000 GT \\
		\bottomrule
	\end{tabularx}
\end{table*}


\section{Bibliography}
\label{sec:bibliography}

We use \lstinline{biblatex} with \lstinline{biber} as the backend for the references. The file with the entries should be loaded before the main document starts with \lstinline|\addbibresource{}|.

There are two main citation commands: for in-line citations use \lstinline|\citet{}| and for citations in parantheses use \lstinline|\citep{}|, see Tab~\ref{tab:citation-commands}.

\begin{lstlisting}
See~\citet{article,unpublished}.
See~\citet{article,phdthesis}.
See~\citep{article}.
\end{lstlisting}

\begin{quote}
\noindent See~\citet{article,unpublished}.

\noindent See~\citet{article,phdthesis}.

\noindent See~\citep{article}.
\end{quote}

\begin{table}[t]
	\begin{center}
		\caption{Some citation commands. Taken from \citet{overleaf2309}.}
		\label{tab:citation-commands} % \label after \caption!!
		\setlength\defaultaddspace{0.5em}
		\begin{tabularx}{\columnwidth}{lX}
			\toprule
			Command & Description \\
			\midrule
			\lstinline{\\citet\{\}} & Textual citation \\
			\lstinline{\\citep\{\}} & Parenthetical citation \\
			\lstinline{\\citet*\{\}} & Same as \lstinline{\\citet} but if there are several authors, all names are printed \\
			\lstinline{\\citep*\{\}} & The same as \lstinline{\\citep} but if there are several authors, all names are printed \\
			\lstinline{\\citeauthor\{\}} & Prints only the name of the authors(s) \\
			\lstinline{\\citeyear\{\}} & Prints only the year of the publication \\

			\bottomrule
		\end{tabularx}
	\end{center}
\end{table}

To manage your references you may want to include a separate file. We give here a template which might work for most cases.

If you want to refer to lecture notes, please use \lstinline{misc}.

\begin{lstlisting}
% references.bib

@article{article,
  author   = "{Doe}, Jane and {Doe}, John",
  title    = "My Wonderful Phoebe Article",
  journal  = "Phoebe",
  year     = 2023,
  volume   = "1",
  number   = "1",
  pages    = "1-5",
}

@book{book,
  author = {{Doe}, John},
  title = {My New Bestseller},
  year = {2023},
  publisher = {University Publishing}
}

@online{online,
  label = {Phoebe},
  title = {My First Online Article},
  url = {https://phoebe.pubpub.org},
  year = {2023},
  urldate = {2023-06-02}
}

@mastersthesis{mastersthesis,
  author  = "{Doe}, Jane",
  title   = "My Master Thesis",
  school  = "My University",
  year    = 2020,
  address = "My Home Town",
  month   = sep
}

@phdthesis{phdthesis,
  author  = "{Doe}, John",
  title   = "My Doctoral Thesis",
  school  = "My University",
  address = "My Home Town",
  year    = 2023,
  month   = jun
}

@misc{misc,
  title        = "Some Miscellaneous Stuff",
  author       = "{Phoebe}",
  howpublished = "\url{https://phoebe.pubpub.org}",
  year         = 2023,
  note         = "Accessed: 2023-07-06"
}

@unpublished{unpublished,
  author = "{Doe}, Jane",
  title  = "Some Yet To Be Published Stuff",
  year   = 2023
}
\end{lstlisting}

%\begin{lstlisting}
%\begingroup
%\RaggedRight
%\printbibliography
%\endgroup
%\end{lstlisting}

\begingroup
\RaggedRight
\printbibliography
\endgroup

\end{document}